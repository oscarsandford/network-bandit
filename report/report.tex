%%%%%%%% ICML 2020 EXAMPLE LATEX SUBMISSION FILE %%%%%%%%%%%%%%%%%
\documentclass{article}

% Recommended, but optional, packages for figures and better typesetting:
\usepackage{microtype}
\usepackage{graphicx}
\usepackage{subfigure}
\usepackage{booktabs} % for professional tables

% hyperref makes hyperlinks in the resulting PDF.
% If your build breaks (sometimes temporarily if a hyperlink spans a page)
% please comment out the following usepackage line and replace
% \usepackage{icml2020} with \usepackage[nohyperref]{icml2020} above.
\usepackage{hyperref}

% Attempt to make hyperref and algorithmic work together better:
\newcommand{\theHalgorithm}{\arabic{algorithm}}

% Use the following line for the initial blind version submitted for review:
% \usepackage{icml2021}
\usepackage[accepted]{icml2021}

% If accepted, instead use the following line for the camera-ready submission:
%\usepackage[accepted]{icml2020}

\usepackage{times}
\usepackage{epsfig}
\usepackage{amsmath}
\usepackage{amssymb}
\usepackage{comment}
\newcommand{\nop}[1]{}

% The \icmltitle you define below is probably too long as a header.
% Therefore, a short form for the running title is supplied here:
\icmltitlerunning{Multi-Armed Bandits for Optimizing New Peers in Peer-to-Peer Networks}

\begin{document}

\twocolumn[
\icmltitle{Multi-Armed Bandits for Optimizing New Peers in Peer-to-Peer Networks}

% It is OKAY to include author information, even for blind
% submissions: the style file will automatically remove it for you
% unless you've provided the [accepted] option to the icml2020
% package.

% List of affiliations: The first argument should be a (short)
% identifier you will use later to specify author affiliations
% Academic affiliations should list Department, University, City, Region, Country
% Industry affiliations should list Company, City, Region, Country

% You can specify symbols, otherwise they are numbered in order.
% Ideally, you should not use this facility. Affiliations will be numbered
% in order of appearance and this is the preferred way.
% \icmlsetsymbol{equal}{*}

\begin{icmlauthorlist}
\icmlauthor{Oscar Sandford}{to}
\icmlauthor{Shawn Nettleton}{to}
\end{icmlauthorlist}

\icmlaffiliation{to}{Department of Computer Science, University of Victoria, Victoria, Canada}

\icmlcorrespondingauthor{Oscar Sandford}{oscarsandford@uvic.ca}
\icmlcorrespondingauthor{Shawn Nettleton}{shawnnettleton@uvic.ca}

% You may provide any keywords that you
% find helpful for describing your paper; these are used to populate
% the "keywords" metadata in the PDF but will not be shown in the document
\icmlkeywords{Machine Learning, ICML}

\vskip 0.3in
]

% this must go after the closing bracket ] following \twocolumn[ ...

% This command actually creates the footnote in the first column
% listing the affiliations and the copyright notice.
% The command takes one argument, which is text to display at the start of the footnote.
% The \icmlEqualContribution command is standard text for equal contribution.
% Remove it (just {}) if you do not need this facility.

\printAffiliationsAndNotice{}  % leave blank if no need to mention equal contribution
%\printAffiliationsAndNotice{\icmlEqualContribution} % otherwise use the standard text.

\begin{abstract}
Write this last (fewer than 300 words). The completed document should be 5-9 pages.
\end{abstract}

\section{Introduction}
Introduction here. Problem, why it is important/interesting, and the plan for the approach.

%-------------------------------------------------------------------------

\section{Related Work}

Example of related work section. Discuss relevant literature. \cite{mab_algos}


%-------------------------------------------------------------------------
\section{Problem Formulation}

Explain the technical bits.


%-------------------------------------------------------------------------
\section{Approach}

Algorithms we will use and develop (e.g. eps-greedy, UCB, and more). Implementation details, pseudocode here.

%----------------------------------------------------------------------------------

\section{Results}

Use of implementation to produce results. Graphs here.

%-------------------------------------------------------------------------------
\section{Discussion}

Discussion on results. Pros and cons of suggested solution compared with existing solutions. 

\section{Conclusion and Future Research}

What we learned. Future work, takeaways.


\bibliography{refs}
\bibliographystyle{icml2021}

\end{document}