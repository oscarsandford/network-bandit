%%%%%%%% ICML 2020 EXAMPLE LATEX SUBMISSION FILE %%%%%%%%%%%%%%%%%
\documentclass{article}

% Recommended, but optional, packages for figures and better typesetting:
\usepackage{microtype}
\usepackage{graphicx}
\usepackage{subfigure}
\usepackage{booktabs} % for professional tables

% hyperref makes hyperlinks in the resulting PDF.
% If your build breaks (sometimes temporarily if a hyperlink spans a page)
% please comment out the following usepackage line and replace
% \usepackage{icml2020} with \usepackage[nohyperref]{icml2020} above.
\usepackage{hyperref}

% Attempt to make hyperref and algorithmic work together better:
\newcommand{\theHalgorithm}{\arabic{algorithm}}

% Use the following line for the initial blind version submitted for review:
% \usepackage{icml2021}
\usepackage[accepted]{icml2021}

% If accepted, instead use the following line for the camera-ready submission:
%\usepackage[accepted]{icml2020}

\usepackage{times}
\usepackage{epsfig}
\usepackage{amsmath}
\usepackage{amssymb}
\usepackage{comment}
\newcommand{\nop}[1]{}

% The \icmltitle you define below is probably too long as a header.
% Therefore, a short form for the running title is supplied here:
\icmltitlerunning{Multi-Armed Bandits for Optimizing New Peers in Peer-to-Peer Networks}

\begin{document}

\twocolumn[
\icmltitle{Multi-Armed Bandits for Optimizing New Peers in Peer-to-Peer Networks}

% It is OKAY to include author information, even for blind
% submissions: the style file will automatically remove it for you
% unless you've provided the [accepted] option to the icml2020
% package.

% List of affiliations: The first argument should be a (short)
% identifier you will use later to specify author affiliations
% Academic affiliations should list Department, University, City, Region, Country
% Industry affiliations should list Company, City, Region, Country

% You can specify symbols, otherwise they are numbered in order.
% Ideally, you should not use this facility. Affiliations will be numbered
% in order of appearance and this is the preferred way.
% \icmlsetsymbol{equal}{*}

\begin{icmlauthorlist}
\icmlauthor{Oscar Sandford}{to}
\icmlauthor{Shawn Nettleton}{to}
\end{icmlauthorlist}

\icmlaffiliation{to}{Department of Computer Science, University of Victoria, Victoria, Canada}

\icmlcorrespondingauthor{Oscar Sandford}{oscarsandford@uvic.ca}
\icmlcorrespondingauthor{Shawn Nettleton}{shawnnettleton@uvic.ca}

% You may provide any keywords that you
% find helpful for describing your paper; these are used to populate
% the "keywords" metadata in the PDF but will not be shown in the document
\icmlkeywords{Machine Learning, ICML}

\vskip 0.3in
]

% this must go after the closing bracket ] following \twocolumn[ ...

% This command actually creates the footnote in the first column
% listing the affiliations and the copyright notice.
% The command takes one argument, which is text to display at the start of the footnote.
% The \icmlEqualContribution command is standard text for equal contribution.
% Remove it (just {}) if you do not need this facility.

\printAffiliationsAndNotice{}  % leave blank if no need to mention equal contribution
%\printAffiliationsAndNotice{\icmlEqualContribution} % otherwise use the standard text.

\begin{abstract}
Write this last (fewer than 300 words). The completed document should be 5-9 pages.
\end{abstract}

%-------------------------------------------------------------------------------
% Problem, why it is important/interesting, and the plan for the approach.
\section{Introduction}
Peer-to-peer computer networks create a unique environment for content distribution wherein the integrity of the system is not compromised by the failure of 
a single, centralized node in the network. According to \cite{p2p_def}, true peer-to-peer systems require peers to be mutually directly accessible (without 
intermediate entities), as well as the network state or quality of service being preserved in the advent of a peer being removed from the network, for any 
reason.

The requirements for peer-to-peer networks in different application domains vary. However, new peers that are directly accessing the server for the first time 
have no information on the network state. New peers therefore cannot be held accountable to preserve the network state and its content if other nodes disconnect. 
It is essential that this new peer is fed the relevant data as fast as possible in order to fulfill both the requirements of a true peer-to-peer environment, as 
well as any necessary quality of service targets. With the added volatility of a dynamic network setting, the rate at which a new peer can be brought "up to speed" 
becomes far more crucial.

In this study, we abstract the new peer scenario described above as a reinforcement learning problem with multi-armed bandits. The multi-armed bandit (MAB) problem 
involves $k$ slot machines (slot machines are sometimes called one-armed bandits) which pay out reward values according to an internal distribution, of which the agent 
cannot know. The goal is to pick a strategy to learn which arms pay out the most in order to to maximize total reward over a set number of rounds \cite{mab_algos}. 

Various algorithms to solve the multi-armed bandit problem are considered, and a select few are implemented in order to evaluate their efficacy against this problem. 
Related literature is surveyed in order to compare our work with solutions to similar problems and verify the validity of our results. Formulation of this challenge 
as a reinforcement learning problem precedes an explanation of the approach and a discussion of the results. First, a survey of related work concerning the application 
of multi-armed bandits to computer networking problems.

%-------------------------------------------------------------------------
% Related work section. Discuss relevant literature.
\section{Related Work}

Multi-armed bandits serve as a useful abstraction for optimization problems that require decision making with reward outcomes that are initially unknown. In a study 
concerning cognitive radio networks \cite{qos_selection_mab}, secondary user (SU) nodes select a single channel for information exchange at one time, with no knowledge 
about channel quality or availability. The authors use a variation on the upper confidence bound (UCB) algorithm, namely QoS-UCB. Their scenario is called "restless", 
meaning that the states of the arms can fluctuate over time, affecting their internal distrubtions and the resulting payouts.

The task of wireless network selection, with the goal of maximizing perceived quality for the end user, is handled by extending the bandit model to be more flexible 
\cite{muMAB_wireless}. In this formulation, the agent can take one of two actions (which can span multiple time steps): measure or use. The difference is that 
measurement allows only evaluation, whereas using adds exploitation. Measuring takes less time than using, which can span a set number of time steps. Results showed 
that the choice of algorithm depended on the payout distributions. Conservative UCB1 is useful when arm rewards are similar, MLI when one arm is clearly better. 
More aggressive algorithms like POKER can lead to low regret but high variability, and are therefore less reliable \cite{muMAB_wireless}.

Anver and Mannor share methods for multiple multi-armed bandit agents, coordinating with each other and learning stochastic network conditions, which may vary between 
users \cite{multiuser_mab}. Their problem formulation is similar to our intentions, but with the agent transmitting instead of receiving. Further, they bound their rewards 
on the interval $[0,1]$. The problem with this reward formulation when it comes to receiving is that, while outward transmission speed or success may be measureably bounded, 
reception rate is not necessarily bounded. In fact, there may be conditions when the end receiver does not have the resources to unpack the transmission packages in time, 
and will become congested. This paper uses techniques to deal with collisions when two or more users transmit in a single channel \cite{multiuser_mab}. In our problem, we 
are only operating with unicast in a channel selected by the requesting peer. A last thing of note is Anver and Mannor's use of UCB in the channel ranking part of their 
algorithm selection.

Another paper surveys resource scheduling with multi-armed bandits in wireless networks \cite{mab_wireless_scheduling_survey}. They mention that $\varepsilon$-greedy, an 
algorithm that balancing exploration and exploitation, has shortcomings in its "pure" randomness, and does not take into account confidence intervals on the reward estimates 
of each arm. UCB exploits this, and also tapers off exploration over time. The authors make the distinction between single- and multi-player multi-armed bandits (SMAB and 
MMAB), where the former involves a single agent operating the bandit selection mechanism. SMAB have applications in our single peer leeching scenario, as well as centralized 
network algorithms. MMAB often involve distributed selection that sacrifices independence for synchronization overhead \cite{mab_wireless_scheduling_survey}.

Similarly Eshcar Hillel \textit{et al}, explore how $k$ players collaborate to identify an $\varepsilon$-optimal arm in a MAB setting, and determine communicating only
once players are able to learn at a rate of $\sqrt{k}$ times faster than a single player \cite{mab_dist_exploration}. This methodology may prove useful as the number of 
peers within the network fluctuates or quality of service changes. 

Multi-armed bandit problems have been gaining significant attraction, however less effort has been devoting to adapting existing bandit algorithms to specific architectures 
such as peer-to-peer network environments \cite{gossip_based_distrivuted_stochastic}. This paper successfully implements the $\varepsilon$-greedy stochastic algorithm to a 
peer-to-peer network environment, scaling with the size of network, achieving a linear speedup in terms of the number of peers and preserving the asymptotic behaviour of the 
standalone version. They also present a heuristic which has a lower network communication cost which may prove helpful with adapting other related works. Scenarios of 
distributed clustering have shown to be promising at solving MAB problems within peer-to-peer networks \cite{dist_clustering_p2p}. There are two setups in particular, the 
first where all peers are solving the same problem and second where a cluster of peers solve the same problem within each cluster. This has shown to achieve an optimal regret 
rate, defined as the expected difference between the reward sum associated with an optimal strategy and the sum of the collected rewards. We hope to adapt these works within 
our approach.

Competition amongst peers is inevitable within the peer-to-peer network environment, especially when peers are trying to stay up to date. Managing this competition can be 
difficult given its somewhat unpredictable nature. Miao Yang \textit{et al}, develop an online learning policy based on top of a MAB framework to deal with peer competition 
and delayed feedback \cite{p2p_offloading_with_delayed_feedback}. However, their work is with relation to fog networks (FN), a decentralized computing infrastructure between 
the data source and the cloud. We aim to utilize some of their understandings while tackling the peer-to-peer network scenario. 

The work done within \cite{p2p_net_sender_scheduling} showcases new considerations for not only optimizing data rate transmissions in wireless peer-to-peer networks but also 
minimizing power consumption. Similar limitations are becoming popular when analyzing the use of graphical processing units (GPUs) such as the work within \cite{gpu_eng}. As 
more domain problems are aspiring for optimal performance it is important to recognize new aspects such as power consumption. 



%-------------------------------------------------------------------------
\section{Problem Formulation}

Consider the setting of a peer-to-peer network wherein a new peer joins with the intent to be brought "up to speed" with the rest of the network as soon as 
possible (i.e. download all the data in the network from other peers). However, the new peer does not know the network speeds of its seeds, just how much data 
it receives over time when it chooses a peer and receives data from them for one time step. The reward is how many bytes received in that time slot. We will 
assume that data packets are UDP datagrams.

We want to be careful about defining the reward, because we want the agent to choose the peer that is transmitting the fastest. However, consider that network 
speeds may change, and the optimal seed to leech from will not always be the best. We call this a "restless" scenario.

This scenario can be solved with a multi-armed bandit approach. Each peer is considered as an "arm" in the multi-armed bandit algorithm. The agent will choose to 
"pull an arm" and receive reward for a certain number of time steps (by default 1). During every time step, the network dynamics shift, and each arm may transition 
to another one of its states (regardless of if it was the arm selected or not). Naturally, this creates greater variance in average reward payout, which serves to 
simulate the noise present in real-world network systems. 

Previous studies \cite{multiuser_mab,gossip_based_distrivuted_stochastic,p2p_offloading_with_delayed_feedback} consider peer-to-peer networks where the peers communicate 
or compete with one another. However, our scenario assumes no information about peers is initially present on the hardware of the new peer, and that transmitting this 
information ahead of the vital data packets should not be a priority. Therefore, the "trial-and-error" methodology of multi-armed bandit agents fits the learning 
requirements under these constrained conditions.


%-------------------------------------------------------------------------
% Algorithms we will use and develop (e.g. eps-greedy, UCB, and more). Implementation details, pseudocode here.
\section{Approach}

% \cite{mab_algos_1} - theoretical guarantees vs practical results of algorithms.

In this section we briefly outline each algorithm considered, and present our approach for generating test scenarios and measuring the algorithms' efficacy against the 
new peer update task within a small set of hyperparameters. The accompanying source code for our work is maintained on 
GitHub\footnote{https://github.com/oscarsandford/network-bandit}.

\subsection{Algorithms}
The most common baseline algorithm applied to multi-armed bandit problems is called $\varepsilon$-greedy. $\varepsilon$-greedy takes a single hyperparameter $\varepsilon$ 
that dictates the probability of exploration (i.e. choosing a random arm from the possible selections), with $1-\varepsilon$ being the probability of choosing the optimal 
arm based on the average rewards so far. We will employ $\varepsilon$-greedy, as well as some of its variants in this study. 

Firstly, $\varepsilon$-first is an $\varepsilon$-greedy strategy where only exploration is done for the first $T \cdot \varepsilon$ rounds, and pure exploitation 
is done during the remaining rounds. $T$ is the number of rounds (also called steps) per run \cite{mab_algos}. This forced exploration means peak rewards will be delayed, 
but more learning is attained as a result.

In addition, $\varepsilon$-decreasing is another $\varepsilon$-greedy variant wherein the initial $\varepsilon$ value $\varepsilon_0$ is decreased over the number of rounds 
completed. More specifically, the probability of exploration at time $t$ is given by $\varepsilon_t = min\{ 1, \frac{\varepsilon_0}{t}\}$ where $\varepsilon > 0$. Values for 
$\varepsilon_0$ are typically not on the interval $[0,1]$, instead values like $1.0$, $5.0$, and $10.0$ are used \cite{mab_algos}.

SoftMax (also called Boltzmann Exploration) makes action decisions based on \emph{probability matching} methods \cite{mab_algos}. Each arm $a$ of $k$ arms holds an associated 
probability $p_a = e^{Q_a/\tau}/\sum_{a'}e^{Q_{a'}/\tau}$ where $Q_a$ is the action value associated with action $a$. The hyperparameter of SoftMax is $\tau$, called the 
\emph{temperature}. It can be varied similar to $\varepsilon$ in $\varepsilon$-greedy.

% More on algos here..

% UCB (and variants)


\subsection{Implementation}
In order to automate the creation of multiple peers, we devised Algorithm~\ref{alg:alg1} which takes as input a number of peers to generate, and three distributions (with 
their associated hyperparameters) from which to sample the number of states a peer will have, and each state's mean and standard deviation reward. Each state requires a 
mean and standard deviation because rewards are generated using a normal distribution. 

Our baseline peer setup for this assessment does not use Algorithm~\ref{alg:alg1}, it is simply a set of 5 single-state peers: $PeerArm(2,1)$, $PeerArm(4,1)$, 
$PeerArm(6,1)$, $PeerArm(8,1)$, $PeerArm(10,1)$. This simple layout removes the simulated dynamism of multi-state peers, but provides a good baseline evaluation for 
each algorithm. Later, we use Algorithm~\ref{alg:alg1} to create greater numbers of peers, each of which is more realistic in their transmission speed (reward) variance.

\begin{algorithm}[tb]
    \caption{create\_peers}
    \label{alg:alg1}
\begin{algorithmic}
    \STATE {\bfseries Input:} number of peers $n$, the distribution for state counts $\mathbb{S}$ and its parameters $s_p$, the distribution for means $\mathbb{M}$ and 
    its parameters $m_p$, and the distribution for standard deviations $\mathbb{D}$ and its parameters $d_p$
    \STATE {\bfseries Output:} a list of $PeerArm$ objects $\Phi$
    \STATE Initialize $\Phi$ as an empty list
    
    \FOR{\_ = 0 {\bfseries to} $n$}
    \STATE $n_s \sim \mathbb{S}(s_p)$
    \IF{$n_s < 1$}
    \STATE $n_s = 1$
    \ENDIF
    \STATE Initialize $\Pi$ as an empty list of means
    \STATE Initialize $\Sigma$ as an empty list of standard deviations

    \FOR{\_ = 0 {\bfseries to} $n_s$}
    \STATE $\pi \sim \mathbb{M}(m_p)$
    \STATE $\sigma \sim \mathbb{D}(d_p)$
    \STATE Append $\pi$ to $\Pi$
    \STATE Append $\sigma$ to $\Sigma$
    \ENDFOR

    \STATE Initialize $T$ as an empty $n_s \times n_s$ transition matrix
    \FOR{i = 0 {\bfseries to} $n_s$}
    \STATE Sample $n_s$ values $\rho_i \sim Uniform(0, 1)$
    \STATE Set each $T_{i,j}$ to $\frac{\rho_{i,j}}{\sum_j\rho_i}$ $\forall j \in [0, n_s)$
    \ENDFOR

    \STATE $\phi = PeerArm(\Pi, \Sigma, T)$
    \STATE Append $\phi$ to $\Phi$
    \ENDFOR
    \STATE Return $\Phi$
\end{algorithmic}
\end{algorithm}

The algorithms discussed prior ($\varepsilon$-greedy, $\varepsilon$-first, $\varepsilon$-decreasing, SoftMax, and UCB) are implemented in Python. POKER was shown 
by \cite{muMAB_wireless} to have low regret but be unreliable due to high variance, due to these problems, time constraints, and the complexity of the pseudocode included 
in the paper, we deigned to not cover POKER in our evaluation. Each algorithm makes use of a generic $BanditEnv$ environment in order to execute actions and reset the 
environment after each run. 

The action value estimates are computed by using a sample-average estimate of action value, with an initial estimate of 0 for each peer. This method does not take 
advantage of the stochastic state-changing Markov process in each peer, as the agent would have to learn the transition matrix for each peer as well. In the interests of 
time and leaving some more on the table for further research, we hold to the sample-average estimate method for this study.

By default the algorithms will take an action for 1 time step, but we have added functionality such that it is possible to take more than 1 action, and the end reward 
for that step is simply the average of rewards received by taking that action for that many time steps. It is important to note that the peer state transitions will 
continue to be active for each time step. That is, the agent cannot "lock" the network state by taking an action for several time steps. 

%----------------------------------------------------------------------------------
% Use of implementation to produce results. Graphs here.
\section{Results}
Each algorithm completes 100 runs, with 10000 steps each. The average reward for each step is averaged across 100 runs, and each of these averages is plotted for 10000 
steps. In the following two subsections, we briefly discuss preliminary assumptions regarding the results, and then present plots containing visually definitive evaluations 
of each algorithm in our test environment.

% What results we expect? Include assumptions and proof sketches.
\subsection{Theoretical Assumptions} 
The preceding literature made $\varepsilon$-greedy out to be an attractive baseline that fares well for many use cases. We had little expectation for the $\varepsilon$-greedy 
variants $\varepsilon$-first and $\varepsilon$-decreasing, and expected them to perform somewhat similarly. Algorithms based around UCB were commonly used in related works, 
and we therefore expected UCB to be stronger than $\varepsilon$-greedy. SoftMax was expected to be eponymously "softer" than $\varepsilon$-greedy.

% Details on experiments and test results.
\subsection{Experimental Results} 
Text here.


%-------------------------------------------------------------------------------
% Discussion on results. Pros and cons of suggested solution compared with existing solutions. 
\section{Discussion}

Text here.

%-------------------------------------------------------------------------------
% What we learned. Future work, takeaways.
\section{Conclusion and Future Research}

Text here.


\bibliography{refs}
\bibliographystyle{icml2021}

\end{document}