%%%%%%%% ICML 2020 EXAMPLE LATEX SUBMISSION FILE %%%%%%%%%%%%%%%%%
\documentclass{article}

% Recommended, but optional, packages for figures and better typesetting:
\usepackage{microtype}
\usepackage{graphicx}
\usepackage{subfigure}
\usepackage{booktabs} % for professional tables

% hyperref makes hyperlinks in the resulting PDF.
% If your build breaks (sometimes temporarily if a hyperlink spans a page)
% please comment out the following usepackage line and replace
% \usepackage{icml2020} with \usepackage[nohyperref]{icml2020} above.
\usepackage{hyperref}

% Attempt to make hyperref and algorithmic work together better:
\newcommand{\theHalgorithm}{\arabic{algorithm}}

% Use the following line for the initial blind version submitted for review:
% \usepackage{icml2021}
\usepackage[accepted]{icml2021}

% If accepted, instead use the following line for the camera-ready submission:
%\usepackage[accepted]{icml2020}

\usepackage{times}
\usepackage{epsfig}
\usepackage{amsmath}
\usepackage{amssymb}
\usepackage{comment}
\newcommand{\nop}[1]{}

% The \icmltitle you define below is probably too long as a header.
% Therefore, a short form for the running title is supplied here:
\icmltitlerunning{Multi-Armed Bandits for Optimizing New Peers in Peer-to-Peer Networks}

\begin{document}

\twocolumn[
\icmltitle{Multi-Armed Bandits for Optimizing New Peers in Peer-to-Peer Networks}

% It is OKAY to include author information, even for blind
% submissions: the style file will automatically remove it for you
% unless you've provided the [accepted] option to the icml2020
% package.

% List of affiliations: The first argument should be a (short)
% identifier you will use later to specify author affiliations
% Academic affiliations should list Department, University, City, Region, Country
% Industry affiliations should list Company, City, Region, Country

% You can specify symbols, otherwise they are numbered in order.
% Ideally, you should not use this facility. Affiliations will be numbered
% in order of appearance and this is the preferred way.
% \icmlsetsymbol{equal}{*}

\begin{icmlauthorlist}
\icmlauthor{Oscar Sandford}{to}
\icmlauthor{Shawn Nettleton}{to}
\end{icmlauthorlist}

\icmlaffiliation{to}{Department of Computer Science, University of Victoria, Victoria, Canada}

\icmlcorrespondingauthor{Oscar Sandford}{oscarsandford@uvic.ca}
\icmlcorrespondingauthor{Shawn Nettleton}{shawnnettleton@uvic.ca}

% You may provide any keywords that you
% find helpful for describing your paper; these are used to populate
% the "keywords" metadata in the PDF but will not be shown in the document
\icmlkeywords{Machine Learning, ICML}

\vskip 0.3in
]

% this must go after the closing bracket ] following \twocolumn[ ...

% This command actually creates the footnote in the first column
% listing the affiliations and the copyright notice.
% The command takes one argument, which is text to display at the start of the footnote.
% The \icmlEqualContribution command is standard text for equal contribution.
% Remove it (just {}) if you do not need this facility.

\printAffiliationsAndNotice{}  % leave blank if no need to mention equal contribution
%\printAffiliationsAndNotice{\icmlEqualContribution} % otherwise use the standard text.

\begin{abstract}
Write this last (fewer than 300 words). The completed document should be 5-9 pages.
\end{abstract}

%-------------------------------------------------------------------------------
% Problem, why it is important/interesting, and the plan for the approach.
\section{Introduction}
Peer-to-peer computer networks create a unique environment for content distribution wherein the integrity of the system is not compromised by the failure of 
a single, centralized node in the network. According to \cite{p2p_def}, true peer-to-peer systems require peers to be mutually directly accessible (without 
intermediate entities), as well as the network state or quality of service being preserved in the advent of a peer being removed from the network, for any 
reason.

The requirements for peer-to-peer networks in different application domains vary. However, new peers that are directly accessing the server for the first time 
have no information on the network state. New peers therefore cannot be held accountable to preserve the network state and its content if other nodes disconnect. 
It is essential that this new peer is fed the relevant data as fast as possible in order to fulfill both the requirements of a true peer-to-peer environment, as 
well as any necessary quality of service targets. With the added volatility of a dynamic network setting, the rate at which a new peer can be brought "up to speed" 
becomes far more crucial.

In this study, we abstract the new peer scenario described above as a reinforcement learning problem with multi-armed bandits. Various algorithms to solve 
the multi-armed bandit problem are considered, and a select few are implemented in order to evaluate their efficacy against this problem. Related literature 
is surveyed in order to compare our work with solutions to similar problems and verify the validity of our results.

%-------------------------------------------------------------------------
% Example of related work section. Discuss relevant literature.
\section{Related Work}

\subsection{Algorithms Used}

\subsection{Domain \& Network Specific}

Text here.


%-------------------------------------------------------------------------
\section{Problem Formulation}

Consider the setting of a peer-to-peer network wherein a new peer joins with the intent to be brought "up to speed" with the rest of the network as soon as 
possible (i.e. download all the data in the network from other peers). However, the new peer does not know the network speeds of its seeds, just how much data 
it receives over time when it chooses a peer and receives data from them for one time step. The reward is how many bytes received in that time slot.

We want to be careful about defining the reward, because we want the agent to choose the peer that is transmitting the fastest. However, consider that network 
speeds may change, and the optimal seed to leech from will not always be the best.


%-------------------------------------------------------------------------
% Algorithms we will use and develop (e.g. eps-greedy, UCB, and more). Implementation details, pseudocode here.
\section{Approach}

Various algorithms will be considered, starting with epsilon-greedy and UCB (upper confidence bound). More complex bandit algorithms are considered as well.

%----------------------------------------------------------------------------------
% Use of implementation to produce results. Graphs here.
\section{Results}
Text here.

% What results we expect? Include assumptions and proof sketches.
\subsection{Theoretical Results} 
Text here.

% Details on experiments and test results.
\subsection{Experiment Results} 
Text here.


%-------------------------------------------------------------------------------
% Discussion on results. Pros and cons of suggested solution compared with existing solutions. 
\section{Discussion}

Text here.

%-------------------------------------------------------------------------------
% What we learned. Future work, takeaways.
\section{Conclusion and Future Research}

Text here.


\bibliography{refs}
\bibliographystyle{icml2021}

\end{document}